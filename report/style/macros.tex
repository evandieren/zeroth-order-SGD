%----------------------------------------------------
% Environments for Mathematics
%----------------------------------------------------

\newcommand{\prf}[1]{
  \begin{quote}
    \begin{proof}
      {#1}
    \end{proof}
  \end{quote}
}

\newtheorem{theorem}{Theorem}[section]
\newtheorem{proposition}{Proposition}[section]
\newtheorem{lemma}{Lemma}[section]
\newtheorem{corollary}{Corollary}[section]
\newtheorem{definition}{Definition}[section]
\newtheorem{assumption}{Assumption}[section]

\newcommand{\thm}[2]{
  \begin{mdframed}
    \begin{theorem}[{#1}]\label{thm:\thetheorem}
      {#2}
    \end{theorem}\noindent
  \end{mdframed}
}

\newcommand{\prp}[2]{
  \begin{mdframed}
    \begin{proposition}[{#1}]\label{prop:\theproposition}
      {#2}
    \end{proposition}\noindent
  \end{mdframed}
}

\newcommand{\lmm}[2]{
  \begin{mdframed}
    \begin{lemma}[{#1}]\label{lemma:\thelemma}
      {#2}
    \end{lemma}\noindent
  \end{mdframed}
}

\newcommand{\cor}[2]{
  \begin{mdframed}
    \begin{corollary}[{#1}]\label{cor:\thecorollary}
      {#2}
    \end{corollary}\noindent
  \end{mdframed}
}

\newcommand{\dft}[2]{
  \begin{mdframed}
    \begin{definition}[{#1}]\label{def:\thedefinition}
      {#2}
    \end{definition}\noindent
  \end{mdframed}
}

\newcommand{\asmpt}[2]{
  \begin{mdframed}
    \begin{assumption}[{#1}]\label{asmpt:\theassumption}
      {#2}
    \end{assumption}\noindent
  \end{mdframed}
}

\newcommand{\qa}[2]{
  \begin{quote}
    \textsc{Question}: {#1} \\
    \textsc{Answer}: {#2}
  \end{quote}
}

\newcommand{\question}[1]{
  \textbf{{#1}}
}

\newcommand{\answer}[1]{
  \textsc{Answer}: {#1}
}

\newcommand{\hint}[1]{
  \textit{Hint: {#1}}
}

\newcommand{\eg}[2]{
  \begin{quote}
    \textsc{Example} - \textsc{{#1}}: {#2}
  \end{quote}
}

\newcommand{\note}[1]{
  \textsc{Note}: {#1}
}

%----------------------------------------------------
% Labelling
%----------------------------------------------------

\newcommand{\todo}{{\color{blue}@TODO }}

% Equation labelling
\numberwithin{equation}{section}
\newcommand{\labeq}{\label{eq:\theequation}}

% Algorithm labelling: https://tex.stackexchange.com/questions/113403/how-to-use-nameref-with-algorithm2e
\makeatletter
\let\original@algocf@latexcaption\algocf@latexcaption
\long\def\algocf@latexcaption#1[#2]{%
  \@ifundefined{NR@gettitle}{%
    \def\@currentlabelname{#2}%
  }{%
    \NR@gettitle{#2}%
  }%
  \original@algocf@latexcaption{#1}[{#2}]%
}
\makeatother

%----------------------------------------------------
% Letters
%----------------------------------------------------

% Blackboard Bold Letters

\newcommand{\Z}{\ensuremath{\mathbb{Z}}}
\newcommand{\C}{\ensuremath{\mathbb{C}}}
\newcommand{\R}{\ensuremath{\mathbb{R}}}
\newcommand{\F}{\ensuremath{\mathbb{F}}}
\newcommand{\N}{\ensuremath{\mathbb{N}}}
\newcommand{\Q}{\ensuremath{\mathbb{Q}}}
\newcommand{\A}{\ensuremath{\mathbb{A}}}
\renewcommand{\P}{\ensuremath{\mathbb{P}}}
\newcommand{\E}{\ensuremath{\mathbb{E}}}
\newcommand{\V}{\ensuremath{\mathbb{V}}}

%----------------------------------------------------
% Other Commands
%----------------------------------------------------

\newcommand{\shrug}[1][]{%
\begin{tikzpicture}[baseline,x=0.8\ht\strutbox,y=0.8\ht\strutbox,line width=0.125ex,#1]
\def\arm{(-2.5,0.95) to (-2,0.95) (-1.9,1) to (-1.5,0) (-1.35,0) to (-0.8,0)};
\draw \arm;
\draw[xscale=-1] \arm;
\def\headpart{(0.6,0) arc[start angle=-40, end angle=40,x radius=0.6,y radius=0.8]};
\draw \headpart;
\draw[xscale=-1] \headpart;
\def\eye{(-0.075,0.15) .. controls (0.02,0) .. (0.075,-0.15)};
\draw[shift={(-0.3,0.8)}] \eye;
\draw[shift={(0,0.85)}] \eye;
% draw mouth
\draw (-0.1,0.2) to [out=15,in=-100] (0.4,0.95); 
\end{tikzpicture}}

\newcommand{\la}{\ensuremath{\left\langle}}
\newcommand{\ra}{\ensuremath{\right\rangle}}

\newcommand{\PP}{\ensuremath{\mathcal{P}}}
\newcommand{\beegO}{\ensuremath{\mathcal{O}}}
\newcommand{\ep}{\varepsilon}
\let\emptyset\varnothing
\newcommand{\diff}[1]{\ensuremath{\ \text{d}{#1}}}
\renewcommand{\vec}[1]{\ensuremath{\mathbf{#1}}}